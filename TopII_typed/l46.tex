\documentclass[a4paper,11pt,english]{article}
\usepackage{.style}

%%%%%%% Title %%%%%%%%%%%%%%%%%%%%%%%%%%%
\title{\textbf{Topology II - Cohomology}}
\author{Tor Gjone}

%%%%%%% Definitions %%%%%%%%%%%%%%%%%%%%%



\newcommand{\diagSquare}[8]{
\begin{tikzpicture}[node distance=2cm, auto]
\node (a)              { $ #5 $ };
\node (b) [right of=a] { $ #6 $ };
\node (c) [below of=a] { $ #7 $ };
\node (d) [right of=c] { $ #8 $ };
\draw[-to] (a) to node { $ #1 $ } (b);
\draw[-to] (a) to node { $ #2 $ } (c);
\draw[-to] (b) to node { $ #3 $ } (d);
\draw[-to] (c) to node { $ #4 $ } (d);
\end{tikzpicture}
}


\newcommand{\congto}{\xrightarrow{\cong}}
\newcommand{\incto}[1][]{ \xhookrightarrow{#1} }
\newcommand{\xto}[1]{\xrightarrow{#1}}


% Projective spaces
\def\RP{\mathbb{RP}}
\def\CP{\mathbb{CP}}
\def\HP{\mathbb{HP}}


% Commen Cathegories
\DeclareMathOperator{\Top}{Top}
\DeclareMathOperator{\Ab}{Ab}
\DeclareMathOperator{\Cat}{Cat}
\DeclareMathOperator{\CAT}{CAT}
\DeclareMathOperator{\Mod}{Mod}
\DeclareMathOperator{\Ring}{Ring}
\DeclareMathOperator{\Group}{Group}


% Homological Algebra
\DeclareMathOperator{\Hom}{Hom}
\DeclareMathOperator{\Tor}{Tor}
\DeclareMathOperator{\Ext}{Ext}


% Common Lie Groups 
\DeclareMathOperator{\GL}{GL}
\DeclareMathOperator{\SU}{SU}
\DeclareMathOperator{\U}{U}
\DeclareMathOperator{\Sp}{Sp}

% Maths Operators
\DeclareMathOperator{\id}{id}



\DeclareMathOperator{\Ker}{Ker}
\DeclareMathOperator{\im}{Im}


% Standard setts.
\def \N {\mathbb{N}}
\def \Z {\mathbb{Z}}
\def \Q {\mathbb{Q}}
\def \R {\mathbb{R}}
\def \C {\mathbb{C}}
\def \H {\mathbb{H}}

\def \E {\mathbb{E}}
\def \Z {\mathbb{Z}}
\def \I {\mathbb{I}}
\def \J {\mathbb{J}}

% Vector calculus.
\newcommand{\dif}[3][]{
	\ensuremath{\frac{d^{#1} {#2}}{d {#3}^{#1}}}}
\newcommand{\pdif}[3][]{
	\ensuremath{\frac{\partial^{#1} {#2}}{\partial {#3}^{#1}}}}

% Vectors and matricies.
\newcommand{\mat}[1]{\begin{matrix} #1 \end{matrix}}
\newcommand{\pmat}[1]{\begin{pmatrix} #1 \end{pmatrix}}
\newcommand{\bmat}[1]{\begin{bmatrix} #1 \end{bmatrix}}

% Add space around the argument.
\newcommand{\qq}[1]{\quad#1\quad}
\newcommand{\q}[1]{\:\:#1\:\:}

% Implications
\newcommand{\la}{\ensuremath{\Longleftarrow}}
\newcommand{\ra}{\ensuremath{\Longrightarrow}}
\newcommand{\lra}{\ensuremath{\Longleftrightarrow}}

\newcommand{\pwf}[1]{\begin{cases} #1 \end{cases}}
\newcommand{\tif}{\text{if}}

% Shorthand
\newcommand{\vphi}{\varphi}
\newcommand{\veps}{\varepsilon}

\newcommand{\<}[1]{\langle #1 \rangle}

% Notation
\newcommand{\wddef}[1]{\underline{#1}}


% Maths Operators
\theoremstyle{plain}
\theoremstyle{definition}
\newtheorem{thrm}{Theorem}[section]
\newtheorem{prop}[thrm]{Proposition}
\newtheorem{corol}[thrm]{Corollary}
\newtheorem{lemma}[thrm]{Lemma}

\newtheorem{defn}[thrm]{Definition}
\newtheorem{exmp}[thrm]{Example}
\newtheorem{clame}[thrm]{Clame}


\theoremstyle{remark}
% \newtheorem{remark}[thrm]{\normalfont\large\textit Remark}
\newtheorem{remark}[thrm]{Remark}
\newtheorem{note}[thrm]{Note}




%%%%%%%% Content %%%%%%%%%%%%%%%%%%%%%%%%
\begin{document}
\mmaketitle



% NOT Complete

\def \map {\text{map}}
\def \mapcts {\map^{\text{cpt}}}

Reminder about homology:
\[ \Top \xrightarrow[\text{singular complex}]{\rho} (\text{simpicial sets}) 
\xrightarrow[\text{linearization}^{C(-,A)}]{} (\text{chain complex}) 
\xrightarrow[n-\text{th homology group}]{}  \Ab. \]

\begin{itemize}
\item For a space $X$, the \underline{singular complex} $\rho(X)$ is the
simplicial set with 
\[ \rho(X)_n = \mapcts(\nabla^n, X) \]
\[ \nabla^n = \text{topological } n-\text{simplex} = \{ (x_0, ..., x_n) \in
\R^{n+1} \q: x_n \ge 0, x_0 + ... + x_n = 1 \} \]
\item For a simplicial set $Y$ and an abelian group, the \wddef{linearization}
is the chain complex $C(Y; A)$ with 
\[ C_(Y;a) = A[Y_n] \quad A-\text{linarization of} Y_n \qquad ( C_n(Y;A) = 0
\quad\text{for } n<0) \]
\item For a chain complex $C$ and $n \in \Z$, the $n$-th homology group
$H_n(C)$ is 
\[ \frac{\ker(d_n : C_n \to C_{n-1})}{\im(d_{n+1}) : C_{n+1} \to C_n}  \]
\end{itemize}


\subsection{Variation : Cohomology}


\begin{defn}
A \wddef{cochain complex} $C$ consists of abelian groups $C^n$ for $n\in\Z$
and homomorphisms $d^n : C^n \to C^{n+1}$ such that 
\[ d^{n+1} \circ d^n = 0 : C^n \to C^{n+2}.  \]
A morphism $f : C \to D$ of cochain complexes (cochain map) consists of
homomorphisms $f^n: C^n \to D^n$ such that $d_D^n \circ f^n = f^{n+1} \circ
d_C^n$ 



\[ \diagSquare{f^n}{d^n}{}{f^{n+1}}{C^n}{D^n}{C^{n+1}}{D^{n+1}}   \]
% \[ \begin{tikzpicture}[node distance=2cm]
% \node (a) {$C^n$}; \node (b) [right of=a] {$D^n$};
% \node (c) [below of=a] {$C^{n+1}$}a \node (d) [right of=c] {$D^{n+1}$}
% \end{tikzpicture} \]


The $n$-th \wddef{cohomology group} of a cochain complex $C$ is
\[ H^nC = \frac{\ker(d^n : C^n \to C^{n+1})}{\im(d^{n-1} : C^{n-1} \to C^n)} \]

A \wddef{cochain homotopy} between two morphisms $f,g: C \to D$ of cocain
complexes consists of homomorphisms 
\[ s^n: C^n \to D^{n-1} \qq{\text{such that }} d^{n-1}\circ s^n + s^{n+1} \circ d^n =f f^n - g^n  \] 
for all $n\in\Z$.

The main tools and properties carry over from chain complexes to cochain
complexes, with essentially the same proofs, such as:

\begin{itemize}
\item a morphism $f: C \to D$ of cochain complexes induces a homomorphism $H^n f
: H^n C \to H^n D$ for cohomology groups by
\[ (H^n f) [x] = [f^n(x)], \quad x \in \ker(d^n : \C^n \to C^{n+1}). \]
%
\item cochain homotopic morphisms $f,g : C \to D$ between cochain complexes
induces the same map in cohomology, ie. $H^n f = H^n g$.
%
\item every short exact sequence of cochain complexes 
\[ 0 \xto{f} B \xto{g} C \to 0 \]
gives rise to a long exact sequence of cohomology groups:
\[ \dots \to H^n A \xto{H^n f} H^n B \xto{H^n g} H^n C \xto{\partial} H^{n+1}(A)
\to \dots \]
where the connecting homomorphism $\partial$ is defined as follows: given $x\in
C^n$ with $d^n(x) = 0$, choose $\tilde{x} \in B^n$ such that $g^n(\tilde{x}) =
x$, then 
\[ g^{n+1} (d_B^n(\tilde{x})) = d_C^n(g^n(\tilde{x})) = d_C^n(x) = 0, \]
so there is a unique $y\in A^{n+1}$ such that $f^{n+1}(y) =
d_B^{n+1}(\tilde{x})$. Set 
\[ \partial [x] = [y] \in H^{n+1}(A). \]
\end{itemize}

\end{defn}



% \makeatletter
% \newcommand\arrowlen[1]{\def\@arrowlen{ #1 }}
% \arrowlen{1em}
% 
% \renewcommand{\xto}[1]{ \xrightarrow{\mathmakebox[\@arrowlen]{#1}} }
% \renewcommand{\mapsto}{ \xmapsto{\mathmakebox[\@arrowlen]{ }} }
% 
% \makeatother


\newdimen\alength 
\alength=22pt

\newenvironment{mappingD}
{
\let\oldxto\xto

%\makeatletter
% \def\@arrowlen{ 2em }


\renewcommand{\xto}[1]{ \xrightarrow{\mathmakebox[\alength]{#1}} }
\renewcommand{\mapsto}{ \xmapsto{\mathmakebox[\alength]{ \the\alength }} }
% \makeatother

\[ \arraycolsep=5pt \def\arraystretch{1.5}
\begin{array}{ccccc} 
}
{ 
\end{array} \] 

\let\xto\oldxto
\undef\oldxto
}

\alength=3em
\begin{mappingD}
A \otimes H_n C &\xto{f\circ g\circ h}& H_n( A\otimes C) &\xto{H^n f}& \Tor(A, H_{n-1}(C)) \\
a\otimes [x] &\mapsto& [a\otimes x] &&
\end{mappingD}


\[ \begin{matrix}
A \otimes H_n C &\xto{f\circ g}& H_n( A\otimes C) &\xto{H^n f}& \Tor(A,
H_{n-1}(C)) \\
a\otimes [x] &\mapsto& [a\otimes x] &&
\end{matrix} \]

\vspace{.3cm}

\[ 
\arraycolsep=5pt \def\arraystretch{1.5}
\begin{array}{ccccc}
A \otimes H_n C &\xto{f\circ g}& H_n( A\otimes C) &\xto{H^n f}& \Tor(A,
H_{n-1}(C)) \\
a\otimes [x] &\mapsto& [a\otimes x] &&
\end{array} 
\]





\end{document}
