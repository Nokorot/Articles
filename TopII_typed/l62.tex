\documentclass[a4paper,11pt,english]{article}
\usepackage{.style}

%%%%%%% Title %%%%%%%%%%%%%%%%%%%%%%%%%%%
\title{\textbf{Topology II - Manifolds}}
\author{Tor Gjone}

%%%%%%% Definitions %%%%%%%%%%%%%%%%%%%%%

% Commen Cathegories
\DeclareMathOperator{\Top}{Top}
\DeclareMathOperator{\Ab}{Ab}
\DeclareMathOperator{\Cat}{Cat}
\DeclareMathOperator{\CAT}{CAT}
\DeclareMathOperator{\Mod}{Mod}
\DeclareMathOperator{\Ring}{Ring}
\DeclareMathOperator{\Group}{Group}


% Homological Algebra
\DeclareMathOperator{\Hom}{Hom}
\DeclareMathOperator{\Tor}{Tor}
\DeclareMathOperator{\Ext}{Ext}


% Maths Operators
\DeclareMathOperator{\id}{id}
\DeclareMathOperator{\GL}{GL}

\DeclareMathOperator{\Ker}{Ker}
\DeclareMathOperator{\im}{Im}


% Standard setts.
\def \N {\mathbb{N}}
\def \Z {\mathbb{Z}}
\def \Q {\mathbb{Q}}
\def \R {\mathbb{R}}
\def \C {\mathbb{C}}

\def \E {\mathbb{E}}
\def \Z {\mathbb{Z}}
\def \I {\mathbb{I}}
\def \J {\mathbb{J}}

% Vector calculus.
\newcommand{\dif}[3][]{
	\ensuremath{\frac{d^{#1} {#2}}{d {#3}^{#1}}}}
\newcommand{\pdif}[3][]{
	\ensuremath{\frac{\partial^{#1} {#2}}{\partial {#3}^{#1}}}}

% Vectors and matricies.
\newcommand{\mat}[1]{\begin{matrix} #1 \end{matrix}}
\newcommand{\pmat}[1]{\begin{pmatrix} #1 \end{pmatrix}}
\newcommand{\bmat}[1]{\begin{bmatrix} #1 \end{bmatrix}}

% Add space around the argument.
\newcommand{\qq}[1]{\quad#1\quad}
\newcommand{\q}[1]{\:\:#1\:\:}

% Implications
\newcommand{\la}{\ensuremath{\Longleftarrow}}
\newcommand{\ra}{\ensuremath{\Longrightarrow}}
\newcommand{\lra}{\ensuremath{\Longleftrightarrow}}

\newcommand{\pwf}[1]{\begin{cases} #1 \end{cases}}
\newcommand{\tif}{\text{if}}

% Shorthand
\newcommand{\vphi}{\varphi}
\newcommand{\veps}{\varepsilon}

\newcommand{\<}[1]{\langle #1 \rangle}

% Notation
\newcommand{\wddef}[1]{\underline{#1}}


% Maths Operators
\theoremstyle{plain}
\theoremstyle{definition}
\newtheorem{them}{Theorem}[section]
\newtheorem{prop}[them]{Proposition}
\newtheorem{corol}[them]{Corollary}
\newtheorem{lemma}[them]{Lemma}

\newtheorem{defn}[them]{Definition}
\newtheorem{exmp}[them]{Example}
\newtheorem{clame}[them]{Clame}


\theoremstyle{remark}
% \newtheorem{remark}[them]{\normalfont\large\textit Remark}
\newtheorem{remark}[them]{Remark}


%%%%%%%% Content %%%%%%%%%%%%%%%%%%%%%%%%
\begin{document}
\mmaketitle

% Complete

\begin{defn}
An \wddef{$m$-manifold} is a Hausdorff space $M$ such that every point of $M$ has an open neighborhood homeomorphic to $\R^m$.
The number $m \ge 0$ is the \wddef{dimension} of $M$.
\end{defn}

\begin{remark}
The empty space is an $m$-manifold for all $m\ge 0$.
For non-empty manifolds, the dimension is intrinsic and can be calculated from the local homology groups:

Let $M$ be a manifold, $x \in M$. Let $U \subset M$ be an open neighborhood of $x$ that admits a homeomorphism $\phi: \R^m \to U$, such that $\phi(0) = x$.

Then:

\[\begin{tikzpicture}[node distance=2cm, auto]
  \node (tl1) {$H_i(M,M\setminus\{x\}; \Z)$};
  \node (tl) [node distance=4.5cm, right of=tl1] {$H_i(U,U\setminus\{x\}; \Z)$};
  \node (tr) [node distance=4.5cm, right of=tl] {$H_i(\R^m,R^m\setminus\{0\}; \Z)$};
  \node (bl1) [below of=tl1] {$\begin{rcases}\Z &\text{if } i=m \\ 0 &\text{if } i\ne m\end{rcases}$};
  \node (bl)  [below of=tl] {$\tilde{H}_{i-1}(S^{m-1}; \Z)$};
  \node (br)  [below of=tr] {$\tilde{H}_{i-1}(\R^{m} \setminus\{0\}; \Z)$};
  \draw[-to] (tl) to node {excision} (tl1);
  \draw[-to] (tr) to node {$\phi_*$} (tl);
  \draw[-to] (tr) to node {$\partial$} (br);
  \draw[-to] (br) to node {inclusion} (bl);
  \draw[-to] (bl) to node {$\cong$} (bl1);
\end{tikzpicture}\]

So the dimensions of $M$ is the dimension in which the local homology is concentrated.
\end{remark}


\begin{remark}
The Hausdorff condition is included to avoid certain pathological examples, such as the "line with double origin":
\[ X = \R \times \{0,1\} / \sim \]
where $(x, 0) \sim (x,1)$ for all $x \ne 0$.
\end{remark}

\begin{exmp}
Open subsets of $\R^m$ are $m$-manifolds.
\end{exmp}

\begin{exmp}
Let $M$ be a Hausdorff space such that every point has an open neighbourhood that is an $m$-manifold.
Then $M$ is an $m$-manifold. In particular, the disjoint union (with disjoint union topology)
of two $m$-manifolds is an $m$-manifold.
\end{exmp} 

\begin{exmp}
Let  $M$ be an $m$-manifold and $N$ and $n$-manifold. Then $M\times N$ (with the product topology)
is an $(m+n)$-manifold. 
\end{exmp}

\begin{exmp}
The $n$-sphere $S^n = \{(x_1, ..., x_{n+1}) \in \R^{n+1} \q: x_1^2 + ... + x_{n+1}^2 = 1 \}$ is an $n$-manifold.
\end{exmp}


For $x =(x_1, ..., x_{n+1}) \in S^n$ let $Y = \{ y \in \R^{n+1} : \<{y,x} = 0\}$ be the orthogonal complement. The \wddef{stereographic projection} is homeomorphism 
\[ p: S^n\setminus\{-x\} \xrightarrow{\cong} Y \cong \R^{n}, \]
defined by 
\[ p(z) = \frac{z - \<{z,x \cdot x}}{1 + \<{z,x}}. \] 


\begin{exmp}
\def\RP{\mathbb{RP}}

The real projective space $\RP^n = S^n / \textbf{antipodal map}$ is an $n$-manifold.
Consider any point $\{x, -x\} \in \RP^n$, choose one of the points $x$. Let 
\[ U = \{ z \in S^n \q: \<{z,x} > 0\} = \text{"hemisphere around x"}. \]
Then the composite
\[ \R^n \cong U \hookrightarrow S^n \xrightarrow{\text{quotient}} \RP^n \]
is a homeomorphism of open neighbourhoods of $\{x,-x\}$.
\end{exmp}

\begin{exmp}

The complex projective space $\CP^n = \{ L \subset \C^{n+1} \q: L \text{ a } 1\text{-dim } \C\text{-subspace of } \C \}$ is a $2n$-manifold. Consider first $L_0 = [0:...:0:1] \in \CP^n$.
Then 
\[ \R^{2n} = \C^n \to \CP^n, \]
defined by
\[ (z_1, ..., z_n) \mapsto [z_1 : ... : z_n : 1] \]
is a homeomorphism onto an open neighbourhood of $L_0$.

If $L\in \CP^n$ is any complex line in $\C^{n+1}$, let $v\in L$ be a non-zero vector, and choose an invertible matrix $A \in \GL_{n+1}(\C)$ such that $A \cdot (0,...,0,1) = v$.
Then
\[ A: \CP^n \to \CP^n; L \mapsto A \cdot L \]
is a self-homomorphism of $\CP^n$ that maps $L_0$ to $\C \cdot v = L$. Since $\CP^n$ is locally homeomorphic to $\R^{2n}$ around $L_0$, is is also locally homeomorphic to $\R^{2n}$ around $L$.
\end{exmp}

\begin{exmp}
The quaternic projective space $\HP^n = \{ L \subset \H^{n+1} \q: L \text{ a } 1\text{-dim left } \H\text{-subspace of } \H \}$.
Similarly as for the complex case, $\HP^n$ is a $4n$-manifold.
\end{exmp}

\end{document}
