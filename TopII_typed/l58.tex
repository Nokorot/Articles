\documentclass[a4paper,11pt,english]{article}
\usepackage{.style}

%%%%%%% Title %%%%%%%%%%%%%%%%%%%%%%%%%%%
\title{\textbf{The topological Künneth theorem in homology}}
\author{Tor Gjone}

%%%%%%% Definitions %%%%%%%%%%%%%%%%%%%%%

% Commen Cathegories
\DeclareMathOperator{\Top}{Top}
\DeclareMathOperator{\Ab}{Ab}
\DeclareMathOperator{\Cat}{Cat}
\DeclareMathOperator{\CAT}{CAT}
\DeclareMathOperator{\Mod}{Mod}
\DeclareMathOperator{\Ring}{Ring}
\DeclareMathOperator{\Group}{Group}


% Homological Algebra
\DeclareMathOperator{\Hom}{Hom}
\DeclareMathOperator{\Tor}{Tor}
\DeclareMathOperator{\Ext}{Ext}


% Maths Operators
\DeclareMathOperator{\id}{id}
\DeclareMathOperator{\GL}{GL}

\DeclareMathOperator{\Ker}{Ker}
\DeclareMathOperator{\im}{Im}


% Standard setts.
\def \N {\mathbb{N}}
\def \Z {\mathbb{Z}}
\def \Q {\mathbb{Q}}
\def \R {\mathbb{R}}
\def \C {\mathbb{C}}

\def \E {\mathbb{E}}
\def \Z {\mathbb{Z}}
\def \I {\mathbb{I}}
\def \J {\mathbb{J}}

% Vector calculus.
\newcommand{\dif}[3][]{
	\ensuremath{\frac{d^{#1} {#2}}{d {#3}^{#1}}}}
\newcommand{\pdif}[3][]{
	\ensuremath{\frac{\partial^{#1} {#2}}{\partial {#3}^{#1}}}}

% Vectors and matricies.
\newcommand{\mat}[1]{\begin{matrix} #1 \end{matrix}}
\newcommand{\pmat}[1]{\begin{pmatrix} #1 \end{pmatrix}}
\newcommand{\bmat}[1]{\begin{bmatrix} #1 \end{bmatrix}}

% Add space around the argument.
\newcommand{\qq}[1]{\quad#1\quad}
\newcommand{\q}[1]{\:\:#1\:\:}

% Implications
\newcommand{\la}{\ensuremath{\Longleftarrow}}
\newcommand{\ra}{\ensuremath{\Longrightarrow}}
\newcommand{\lra}{\ensuremath{\Longleftrightarrow}}

\newcommand{\pwf}[1]{\begin{cases} #1 \end{cases}}
\newcommand{\tif}{\text{if}}

% Shorthand
\newcommand{\vphi}{\varphi}
\newcommand{\veps}{\varepsilon}

\newcommand{\<}[1]{\langle #1 \rangle}

% Notation
\newcommand{\wddef}[1]{\underline{#1}}


% Maths Operators
\theoremstyle{plain}
\theoremstyle{definition}
\newtheorem{them}{Theorem}[section]
\newtheorem{prop}[them]{Proposition}
\newtheorem{corol}[them]{Corollary}
\newtheorem{lemma}[them]{Lemma}

\newtheorem{defn}[them]{Definition}
\newtheorem{exmp}[them]{Example}
\newtheorem{clame}[them]{Clame}


\theoremstyle{remark}
% \newtheorem{remark}[them]{\normalfont\large\textit Remark}
\newtheorem{remark}[them]{Remark}


%%%%%%%% Content %%%%%%%%%%%%%%%%%%%%%%%%
\begin{document}
\mmaketitle

\section{Construction (Exterior pairing)}
Let $R$ be a commutative ring and $X$ and $Y$ simplical sets. The \wddef{homological exterior pairing}
\begin{align*}
& \times : && H_p(X; R) \otimes_R H_q(Y; R) = H_p(C_*(X; R)) \otimes_R H_q(C_*(Y; R)) \\ 
& \xrightarrow[{[x]\otimes[y] \mapsto [x\otimes y]}]{\Phi}  && H_{p+q}(C_*(X;R)\otimes_R C_*(Y;R))  \\
& \xrightarrow{\qq{H_{p+q}(\nabla)}} && H_{p+q}(C_*(X\times Y; R)) = H_{p+q}(X\times Y;R).
\end{align*} 

For two spaces $A$ and $B$, the \wddef{external homology pairing} is the composite

\begin{align*}
& \times : 
&& H_p(A; R) \otimes_R H_q(B; R) = H_{p} (\rho(A); R) \otimes_R H_q(\rho(B);R) \\ 
& \xrightarrow{\qq{\qq\times}}  
&&  H_{p+q}(\rho(A)\times\rho(B); R)  \\
& \xrightarrow{H_{p+q}(c^{-1};R)}
&& H_{p+q}(\rho(A\times B); R) = H_{p+q}(A\times B;R).
\end{align*} 

where $c = (\rho(p_1), \rho(p_2)): \rho(A \times B) \xrightarrow[\cong] \rho(A)\times \rho(B)$ is the canonical isomorphism of simplical sets, where $p_1 : A\times B \to A$ and $p_2 : A\times B \to B$ are the canonical projections.

\begin{them}
Let $R$ be a field and $X$ and $Y$ be spaces or simpical spaces. Then the exterioir homology pairing provides a natural isomorphism of $R$-vector spaces
\[ \bigoplus_{p+q=n} H_p(X;R)\otimes_R H_q(Y;R) \to H_n(X\times Y; R) \]
\end{them}

\begin{proof}
case of siplical sets: the map in question is the composite of two isomorphisms: 
\[ \bigoplus_{p+q=n} H_p(C_*(X; R)) \otimes_R H_q(C_*(Y; R)) \xrightarrow{\Phi} 
H_n(C_*(X;R)\otimes_R C_*(Y;R)) \xrightarrow{H_n(\nabla)}
H_n(C_*(X\times Y; R)), \]
where the first map is an isomorphism by the algebraic Künneth theorem and the second is an isomorphism by the Elenberg-Zilber theorem.
\end{proof}

Simular arguments show:

\begin{them}
Let $X$ and $Y$ be spaces or siplical sets. Then the singular exterios homology pairing participate in a natural short exact sequence of abelian groups: 
\[ 0 \to \bigoplus_{p+q=n} H_p(X;\Z) \otimes H_q(Y;\Z) \xrightarrow{\times} H_n(X\times Y; \Z) \to \bigoplus_{p+q=n} \Tor(H_p(X,\Z), H_q(Y,\Z)) \to 0 \]

The sequence splits (but the spliting cannot be chosen naturaly).
\end{them}


\end{document}
