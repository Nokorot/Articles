\documentclass[a4paper,11pt,english]{article}
\usepackage{.style}

%%%%%%% Title %%%%%%%%%%%%%%%%%%%%%%%%%%%
\title{\textbf{Topology II - Cohomology}}
\author{Tor Gjone}

%%%%%%% Definitions %%%%%%%%%%%%%%%%%%%%%

% Commen Cathegories
\DeclareMathOperator{\Top}{Top}
\DeclareMathOperator{\Ab}{Ab}
\DeclareMathOperator{\Cat}{Cat}
\DeclareMathOperator{\CAT}{CAT}
\DeclareMathOperator{\Mod}{Mod}
\DeclareMathOperator{\Ring}{Ring}
\DeclareMathOperator{\Group}{Group}


% Homological Algebra
\DeclareMathOperator{\Hom}{Hom}
\DeclareMathOperator{\Tor}{Tor}
\DeclareMathOperator{\Ext}{Ext}


% Maths Operators
\DeclareMathOperator{\id}{id}
\DeclareMathOperator{\GL}{GL}

\DeclareMathOperator{\Ker}{Ker}
\DeclareMathOperator{\im}{Im}


% Standard setts.
\def \N {\mathbb{N}}
\def \Z {\mathbb{Z}}
\def \Q {\mathbb{Q}}
\def \R {\mathbb{R}}
\def \C {\mathbb{C}}

\def \E {\mathbb{E}}
\def \Z {\mathbb{Z}}
\def \I {\mathbb{I}}
\def \J {\mathbb{J}}

% Vector calculus.
\newcommand{\dif}[3][]{
	\ensuremath{\frac{d^{#1} {#2}}{d {#3}^{#1}}}}
\newcommand{\pdif}[3][]{
	\ensuremath{\frac{\partial^{#1} {#2}}{\partial {#3}^{#1}}}}

% Vectors and matricies.
\newcommand{\mat}[1]{\begin{matrix} #1 \end{matrix}}
\newcommand{\pmat}[1]{\begin{pmatrix} #1 \end{pmatrix}}
\newcommand{\bmat}[1]{\begin{bmatrix} #1 \end{bmatrix}}

% Add space around the argument.
\newcommand{\qq}[1]{\quad#1\quad}
\newcommand{\q}[1]{\:\:#1\:\:}

% Implications
\newcommand{\la}{\ensuremath{\Longleftarrow}}
\newcommand{\ra}{\ensuremath{\Longrightarrow}}
\newcommand{\lra}{\ensuremath{\Longleftrightarrow}}

\newcommand{\pwf}[1]{\begin{cases} #1 \end{cases}}
\newcommand{\tif}{\text{if}}

% Shorthand
\newcommand{\vphi}{\varphi}
\newcommand{\veps}{\varepsilon}

\newcommand{\<}[1]{\langle #1 \rangle}

% Notation
\newcommand{\wddef}[1]{\underline{#1}}


% Maths Operators
\theoremstyle{plain}
\theoremstyle{definition}
\newtheorem{them}{Theorem}[section]
\newtheorem{prop}[them]{Proposition}
\newtheorem{corol}[them]{Corollary}
\newtheorem{lemma}[them]{Lemma}

\newtheorem{defn}[them]{Definition}
\newtheorem{exmp}[them]{Example}
\newtheorem{clame}[them]{Clame}


\theoremstyle{remark}
% \newtheorem{remark}[them]{\normalfont\large\textit Remark}
\newtheorem{remark}[them]{Remark}


\def\tM{\widetilde{M}}
\def\oD{\mathring{D}}

\usetikzlibrary{positioning}
%%%%%%%% Content %%%%%%%%%%%%%%%%%%%%%%%%
\begin{document}
\mmaketitle

% Complete

Notation: 
\begin{itemize}
\item Coefficients for homology will always be $\Z$, omitted from the notation.
\item If $X$is a space and $Y\subset X$, we write $H_n(X|Y) = H_n(X,X\setminus
Y; \Z)$ "local homology at $Y$".

Note: if $Y \subset U \subset X$ and $U$ is a neighbourhood of $Y$, then
excision provides an isomorphism 
\[ H_n(U|Y) = H_n(U,U\setminus Y) \congto H_n(X,X\setminus Y) = H_n(X|Y). \]
\end{itemize}

For $n$-manifold $M$ and $x\in M$, 
\[ H_n(M|x) := H_n(M,M\setminus \{x\}; \Z) \cong \Z. \]
A \wddef{local orientation} of $M$ at $x$ is a generator of $H_n(M|x)$. There
are exactly two local orientations.

Heuristically:  an orientation of $M$is a "continues choice" of local
orientations.

\underline{Construction: (The orientation covering)}

Let $M$ be an $n$-manifold. Set
\[ \tM = \{ (x,\mu) \q: x \in M, \mu\in H_n(M|x) \text{ a local
orientation} \}. \]

The map $p : \tM \to M$, defined by $p(x,\mu) = x$ is surjective and every
point has exactly two pre-images.

We endow $\tM$ with a topology that makes $p$ into a two fold covering.
A subset $B$ of $M$ is a local ball if $B$ is open and there is a homeomorphism
$\phi : \R^n \to U$, for some open neighbourhood $U$ of $B$, such that
$\phi(\oD^n) = B$, where $\oD^n = \{ x\in\R^n \q: |x| < 1 \}$.


Note: the inclusion $M\setminus B \to M\setminus \{ x \}$ is a homotopy
equivalence for all $x\in B$. So it induces an isomorphism 
\[ r^B_x : H_n(M|B) \congto H_n(M|x) \cong \Z. \]

We let $\mu \in H_n(M|B)$ be a generator and define
\[ U(B,\mu) = \{ (x,r^B_x(\mu)) \q: x\in B \} \subseteq \tM. \]

\begin{thrm}
Let $M$ be an $n$-manifold.
\begin{enumerate}
\item As $(B,\mu)$ varies over all local balls $B$ and all generators of
$H_n(M|B)$, the sets $U(B,\mu)$ form the basis of a topology on $\tM$.

\item For this topology, the map $p: \tM \to M$, defined by $p(x,\nu) = x$, is a
twofold covering map, the \wddef{orientation covering} $M$.

\item $\tM$ is an $n$-manifold.
\end{enumerate}
\end{thrm}

\begin{proof}
\begin{enumerate}
\item We show that $U(B, \mu) \cap U(B',\mu')$ is a union of basis sets.

Let $(x,\nu) \in U(B,\mu) \cap U(B',\mu')$. Then $x\in B\cap B'$ and $\nu =
r^B_x(\mu) = r^{B'}_x(\mu')$. We choose another local ball $B''$ with $x \in B''
\subseteq B\cap B'$.

We obtain a diagram of local homology groups

\[ 
\begin{tikzpicture}[align=center, node distance=3cm, auto]
\node (tl)                      { $H_n(M|B)$ };
\node (ma) [below right=.5cm and 2cm of tl]  { $H_n(M|B\cap B')$ };
\node (mb) [right of=ma]        { $H_n(M|B'')$ };
\node (mc) [right of=mb]        { $H_n(M|x)$ };
\node (bl) [below=1.5cm of tl]   { $H_n(M|B')$ };
\draw[-to] (tl) to node {} (ma);
\draw[-to] (bl) to node {} (ma);
\draw[-to] (ma) to node {} (mb);
\draw[-to] (mb) to node {} (mc);
\draw[-to] (bl) to [out=0, in=200] node {} (mc);
\draw[-to] (bl) to [out=0, in=200] node {} (mb);
\draw[-to] (tl) to [out=0, in=170] node {} (mc);
\draw[-to] (tl) to [out=0, in=170] node {} (mb);
\end{tikzpicture}
\]
Where all the maps are isomorphisms. Since $\mu$ and $\mu'$ restrict to the same
generator of $H_n(M|x)$, they also restrict to the same generator $\mu'' =
r^B_{B''}(\mu) = r^{B'}_{B''}(\mu')$ of
$H_n(M|B'')$. 

Hence:
\[ (x,\nu) \in U(B'', \mu'') \subseteq U(B,\mu) \cap U(B',\mu'). \]
So the sets $U(B,\mu)$ form a basis for a topology on $\tM$.

\item
Because $M$ is an $n$-manifold, the local balls form a basis of the topology of
$M$. Moreover $p^{-1}(B) = U(B,\mu) \dot{\cup} U(B,-\mu)$ where $\pm \mu$ are
the two generators of $H_n(M|B)$.

So $p$ is continues. Morover, the restriction $p|_{U(B,\mu)} : U(B,\mu) \to B$
is a bijective continues map. The map is also open (and hence a homeomorphism)
because a basis of the subspace topology of $U(B,\mu)$ is given by the sets
$U(B'',\mu'')$ for local balls $B'' \subseteq B$ and $\mu'' = r^B_{B''}(\mu)$.
Because $p(U(B'',\mu'')) = B''$is open in $M$, the restriction of $p$ to
$U(B,\mu)$ is an open map.

So $p^{-1} (B) \cong B \coprod B$ is homeomorphic in a way that matches $p$with
the local map $B\coprod B \to B$. So $p$ is a twofold covering map.

\item 
By design, every point $(x,\nu) \in \tM$ has an open neighbourhood $U(B,\mu)$,
which is homeomorphic to $B \cong \oD^n \cong \R^n$. So $\tM$ is locally
euclidean of dimension $n$. Since $M$ is Hausdorff and $p: \tM \to M$ a
covering, $\tM$ is Hausdorff.
\end{enumerate}
\end{proof}

\begin{defn}
An \wddef{orientation} of an $n$-manifold $M$ is a continues section $s: M \to
\tM$ of the covering $p: \tM \to M$. The manifold $M$ is \wddef{orientable} is
there exists an orientation of $M$.
\end{defn}

\begin{remark}
Because manifolds are locally euclidean, their path components are open. So
manifolds are the topological disjoint union of their path components. For many
purposes one can restrict to connected manifolds by considering each path
component separately.
\end{remark}

\begin{corol}
A connected orientable manifold has exactly 2 orientations. An orientable
manifold with $n$ components has $2^n$ orientations.
\end{corol}

\begin{proof}
If $M$ is orientable, then $\tM \cong M\coprod M$, taking $p : \tM \to M$ to the
be fold map. So there are exactly two continues sections if $M$ is connected. In
general, you can independently choose an orientation of each path component.
\end{proof}

\begin{note}
If $M$ is connected, $p: \tM \to M$ is a product cover $\lra$ there is a continues
section of $p$, $\lra$ $M$ is orientable.
\end{note}

\begin{corol}
Let $M$ be a connected $n$-manifold such that for some (hence ant) $x\in M$, the
group $\pi_1(M, x)$ does \underline{not} have a subgroup of index 2. Then $M$is
orientable. In particular, all simply connected manifolds are orientable.
\end{corol}

\begin{proof}
Let $\tilde{x} \in \tM$ be any point over $x$. If $M$ is not orientable, then
$p:\tM \to M$ is not a product cover, and $\tM$ would be connected. So
$p_* : \pi_1(\tM,\tilde{x}) \to \pi_1(M,x)$ is an injective group homomorphism
whose image has index $2$ in $\pi_1(M,x)$. This contradicts the hypothesis, so
$M$ is orientable.
\end{proof}

\begin{exmp}
$S^n$ is simply connected for $n \ge 2$, and hence orientable. For all $n\ge 1$,
$\CP^n$ and $\HP^n$ are simply connected, and hence orientable.
\end{exmp}

\begin{exmp}
Let $M$ be an $n$-manifold that also admits the structure of a topological
group, i.e. a gorup structure such that the mulriplication and iverse maps are
continues in the given topology. Then $M$ is orientable. Examples: $S^1$,
$\O(n)$, $\U(n)$, $\Sp(n)$ and $\SU(n)$.
\end{exmp}

\begin{proof}
Let $m: M\times M \to M$ be the group structure and let $e\in M$ be the natural
element. We choose a local orientation $\mu_0 \in H_n(M|e)$. For any $x\in M$,
the map $m(x,-) : M \to M$ is a homeomorphism, with inverse $m(x^{-1}, -)$, that
sends $e$ to $m$. So it induces an isomorphism of local homology groups
\[ m(x, -)_* : H_(M|e) \congto H_n(M|x); \quad \mu_0 \mapsto \mu_x. \]

We define $s:M\to \tM$ by $s(x) = \mu_x = m(x,-)_* (\mu_0)$, this is continues
and hence an orientation of $M$.
\end{proof}

\newcommand{\tx}{\tilde{x}}

\begin{prop}
Let $M$ be an $n$-manifold
\begin{enumerate}
\item[(i)] The manifold $\tM$ is orientable and the map $\tau : \tM \to \tM$,
$\tau(x,\nu) = (x,-\nu)$ reverses the local orientation of $\tM$.
\item[(ii)] Suppose that $q: N \to M$ is a 2-fold covering and $N$ an orientable
manifold. Moreover, suppose that the non-identity dichtransformation $\tau: N
\to N$ reverses the local orientations. Then $q: N \to M$ is isomorphic, as a
covering, to $p: \tM \to M$.
\end{enumerate}
\end{prop}

\begin{proof}
\begin{enumerate}
\item[(i)] Let $\tx = (x,\mu) \in \tM$. Since $p: \tM \to M$ is a local
homeomorphism, it induces an isomorphism 
\[ p_0 : H_n(\tM,\tx) \congto H_n(M|x) = H_n(M|p(\tx)). \]
We let $p_*^{-1} (\mu)$ be the "tautological" local orientation of $\tM$ at
$\tx$. This defines a continues (!) map 
\[ \tM \to \widetilde{\tM}; \quad \tx = (x,\mu) \mapsto (\tx,p_*^{-1}(\mu)), \]
hence an orientation of $\tM$.


\begin{align*}
\tau_* : H_n(\tM | \tx) &\to H_n(\tM| \tau(\tx)) \\
(\tx, p_*^{-1}) &\mapsto (\tau(\tx), \tau_*(p_*^{-1}(\mu))) \\
&= ((x,-\mu), p_*^{-1}(\mu)) \\
&= ( (x,-\mu), -p_*^{-1}(-\mu)) \\
&\ne ((x,-\mu), p_*(-\mu))
\end{align*}
So $\tau$ reverts the local orientation of $\tM$.

\item[(ii)] Let $q:N \to M$ be any 2-fold covering such that $N$ is orientable
and $\tau : N\to N$ is orientation reverting. Let $\{ \mu_y \}_{y\inN}$ be an
oreintation of $N$. We define $f: N \to \tM$ by $f(y) = (q(y, q_*(\mu_y)) \in
\tM)$, where $q_* : H_n(N,y) \congto H_n(M, q(y))$.
The continuity of the local orientations $\{\mu_y\}$ ipmiles the continuity of
$f(!)$. Because $\tau : N\to N$ reverses the orientation, $f$ is compatible with
the non-trivial dich transformations:
\begin{align*}
f(\tau y) &= (q(y), q_*(\mu_{\tau y})) \\
&= (q(y), q_*(-\tau_*|\mu_y)), \quad \tau: N\to N \text{ reverts the
orientation} \\
&= (q(y), -q_*(\mu_y)), \qquad q\tau = q \\
&= \tau(q(y), q_*(\mu_y)) \\
&= \tau(f(y))
\end{align*}

So $f$is a continues bijection between two fold coverings over the same base
$M$, so $f$ is also open, and hence an isomorphism of coverings.
\end{enumerate}
\end{proof}


\begin{exmp}
The antipital map $A : S^n \to S^n$, defined by $A(x) = -x$, has degree
$(-1)^{n+1}$.
LEt $e \in H_n(S^n; \Z)$ be any generator and orient $S^n$ by the image of $e$
in $H_n(S^n|x)$ for all $x \in S^n$. 
So $A$is orientation reversing if and only if $n$ is even.
The projection $q:S^n \to \RP^n$, $q(x) = \R \cdot x$, is a 2-fold covering with
orientable total space. 

If $n$ is even, the non-identity dechtransformation $A$ reverses the
orientation. So for $n$ even, $q: S^n \to \RP^n$ is isomorphic to the
orientatin covering of $\RP^n$. So for all $n$ even, $\RP^n$ is
\underline{not} orientable.
\end{exmp}

\begin{exmp}
For $n$ odd, $\RP^n$ is orientable. We construct an orientation of $\RP^n$ by
choosing a generator $e \in H_n(S^n; \Z)$ and define the local orientation at
$\R \cdot x \in \RP^n$  as the image of $e$ under the isomorphism.
\[ H_n(S^n; \Z) \congto H_n(S^n|x) \xrightarrow[q]{\cong} H_n(\RP^n| \R\cdot x);
e \mapsto \mu_{\R\cdot x} \]

Because the antipital map has degree $+$, this yields the same orientation for
$x$ and $-x$. In short all the local orientations arise from one class in
$H_n(\RP^n;\Z)$, so they vary continuesly in $\R x \in \RP^n$.
\end{exmp}


\end{document}
