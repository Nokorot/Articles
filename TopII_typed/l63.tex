\documentclass[a4paper,11pt,english]{article}
\usepackage{.style}

%%%%%%% Title %%%%%%%%%%%%%%%%%%%%%%%%%%%
\title{\textbf{Topology II - Stiefel manifolds}}
\author{Tor Gjone}

%%%%%%% Definitions %%%%%%%%%%%%%%%%%%%%%



\newcommand{\diagSquare}[8]{
\begin{tikzpicture}[node distance=2cm, auto]
\node (a)              { $ #5 $ };
\node (b) [right of=a] { $ #6 $ };
\node (c) [below of=a] { $ #7 $ };
\node (d) [right of=c] { $ #8 $ };
\draw[-to] (a) to node { $ #1 $ } (b);
\draw[-to] (a) to node { $ #2 $ } (c);
\draw[-to] (b) to node { $ #3 $ } (d);
\draw[-to] (c) to node { $ #4 $ } (d);
\end{tikzpicture}
}


\newcommand{\congto}{\xrightarrow{\cong}}
\newcommand{\incto}[1][]{ \xhookrightarrow{#1} }
\newcommand{\xto}[1]{\xrightarrow{#1}}


% Projective spaces
\def\RP{\mathbb{RP}}
\def\CP{\mathbb{CP}}
\def\HP{\mathbb{HP}}


% Commen Cathegories
\DeclareMathOperator{\Top}{Top}
\DeclareMathOperator{\Ab}{Ab}
\DeclareMathOperator{\Cat}{Cat}
\DeclareMathOperator{\CAT}{CAT}
\DeclareMathOperator{\Mod}{Mod}
\DeclareMathOperator{\Ring}{Ring}
\DeclareMathOperator{\Group}{Group}


% Homological Algebra
\DeclareMathOperator{\Hom}{Hom}
\DeclareMathOperator{\Tor}{Tor}
\DeclareMathOperator{\Ext}{Ext}


% Common Lie Groups 
\DeclareMathOperator{\GL}{GL}
\DeclareMathOperator{\SU}{SU}
\DeclareMathOperator{\U}{U}
\DeclareMathOperator{\Sp}{Sp}

% Maths Operators
\DeclareMathOperator{\id}{id}



\DeclareMathOperator{\Ker}{Ker}
\DeclareMathOperator{\im}{Im}


% Standard setts.
\def \N {\mathbb{N}}
\def \Z {\mathbb{Z}}
\def \Q {\mathbb{Q}}
\def \R {\mathbb{R}}
\def \C {\mathbb{C}}
\def \H {\mathbb{H}}

\def \E {\mathbb{E}}
\def \Z {\mathbb{Z}}
\def \I {\mathbb{I}}
\def \J {\mathbb{J}}

% Vector calculus.
\newcommand{\dif}[3][]{
	\ensuremath{\frac{d^{#1} {#2}}{d {#3}^{#1}}}}
\newcommand{\pdif}[3][]{
	\ensuremath{\frac{\partial^{#1} {#2}}{\partial {#3}^{#1}}}}

% Vectors and matricies.
\newcommand{\mat}[1]{\begin{matrix} #1 \end{matrix}}
\newcommand{\pmat}[1]{\begin{pmatrix} #1 \end{pmatrix}}
\newcommand{\bmat}[1]{\begin{bmatrix} #1 \end{bmatrix}}

% Add space around the argument.
\newcommand{\qq}[1]{\quad#1\quad}
\newcommand{\q}[1]{\:\:#1\:\:}

% Implications
\newcommand{\la}{\ensuremath{\Longleftarrow}}
\newcommand{\ra}{\ensuremath{\Longrightarrow}}
\newcommand{\lra}{\ensuremath{\Longleftrightarrow}}

\newcommand{\pwf}[1]{\begin{cases} #1 \end{cases}}
\newcommand{\tif}{\text{if}}

% Shorthand
\newcommand{\vphi}{\varphi}
\newcommand{\veps}{\varepsilon}

\newcommand{\<}[1]{\langle #1 \rangle}

% Notation
\newcommand{\wddef}[1]{\underline{#1}}


% Maths Operators
\theoremstyle{plain}
\theoremstyle{definition}
\newtheorem{thrm}{Theorem}[section]
\newtheorem{prop}[thrm]{Proposition}
\newtheorem{corol}[thrm]{Corollary}
\newtheorem{lemma}[thrm]{Lemma}

\newtheorem{defn}[thrm]{Definition}
\newtheorem{exmp}[thrm]{Example}
\newtheorem{clame}[thrm]{Clame}


\theoremstyle{remark}
% \newtheorem{remark}[thrm]{\normalfont\large\textit Remark}
\newtheorem{remark}[thrm]{Remark}
\newtheorem{note}[thrm]{Note}


\newcommand{\textif}{\text{if }}


\usepackage{tikz}

\usetikzlibrary{arrows.meta}

%%%%%%%% Content %%%%%%%%%%%%%%%%%%%%%%%%
\begin{document}
\mmaketitle

% Complete

Let $0 \le k \le n$. The \wddef{Stiefel manifold} is defined by 
\begin{align}
V_{k,n} &= \left\{
(v_1, \dots, b_k) \in (\R^n)^k \q: \<{v_i,v_j} = \begin{cases}1 & \textif i=j \\ 0
& \textif i\ne j \end{cases} \right\} \\
&= \text{space of orthogonal } k\text{-frames in } \R^n 
\end{align}

$V_{k,n}$ comes with the subspace topology of $(\R^n)^k$ ; since
$V_{k,n} \subset (S^{n-1})^4$ is a closed subset, $V_{k,n}$is compact.

\begin{exmp}
\begin{align*}
V_{0,n} &= \{ \emptyset \} \text{ is a one-pint space.} \\
V_{1,n} &= S^{n-1} \\
F: V_{n,n} &\xleftrightarrow{\cong} O(n) : G \\
\end{align*}
where $F$ maps $(v_1, ..., v_n)$ to the matrix with colomns $v_i$ and $G$ maps
$A$ to $(A e_1, ..., A e_n)$ where $e_i$ is the unit vecotr with a $0$ in the
$i$-th entry and $0$'s elsewhere.

The map $F : SO(n) \to V_{n-1,n}$ defined by $A \mapsto (A e_1, ..., A
e_{n-1})$, is a continues bijection beween compact Housdorff spaces, and hence a
homeomorphism.

Bijectivity: Let $(v_1, ..., v_{n-1})$ be an $(n-1)$-frame in $\R^n$, then the
orthonogal complement of the span of $v_1,...,v_{n-1}$ is 1-dim. So there are
exactly 2 unit vectors in this complement. Exactly one of these makes
$(v_1, ... ,v_{n-1},v_n)$ into an orthogonal basis of determinant $+1$.
\end{exmp}
    
\begin{prop}
The space $V_{k,n}$is a manifold of dimension
\[ (n-1) + (n-2) + ... + (n-k) = nk - \frac{k(k+1)}{2}. \]
\end{prop}

\begin{proof}
By induction on $k$. For $k=0$, $V_{0,n} = \{ \emptyset \}$ is a 0-manifold and for
$k=1$, $V_{1,n} = S^{n-1}$ is a $(n-1)$-manifold.

Now suppose $k \ge 2$. We consider the map $\phi : S_+^{n-1} \to O(n)$, where
\[ S_+^{n-1} = \{ w \in S^{n-1} \q: w_1 > 0 \}, \]
defined be the composition 

\[
\begin{tikzpicture}[node distance=2cm, auto]
  \node (tl) {$S_+^{n-1}$};
  \node (tm) [node distance=4.5cm, right of=tl] {$\GL_n(\R)$} ;
  \node (tr) [node distance=5cm, right of=tm] {$O(n)$};
  \node (bl) [below of=tl] {$O(n)$};
  \node (bm) [below of=tm] {$
        \pmat{ w_1 & 0 & \dots & 0 \\ 
              w_2 & 1 &  & 0 \\
           \vdots &   & \ddots &  &  \\
              w_n & 0 &  & 1
        }$};
  \node (br) [below of=tr] {};
  \draw[-to] (tl) to node {} (tm);
  \draw[-to] (tm) to node {Gram-Schmidt Orth.} (tr);
  \draw[|->] (bl) to node {$\cong$} (bm);
\end{tikzpicture}
\]

Properties of $\psi$:
\begin{itemize}
\item $\psi$ is continues 
\item $\psi(e_1) = \psi(1,0,..,0) = E_n = $ identity matrix
\item $\psi(w) \cdot e_1 = w$ for all $w \in S_+^{n-1}$
\end{itemize}

Warning: There is not continues map $\psi : S^{n-1} \to O(n)$ such that
$\psi(w) \cdot w = w$ for all $w\in S^{n-1}$.

We define $U = \{ (v_1 ,...,v_k) \in V_{k,n} \q: v_1 \in S_+^{n-1} \}$; this is
an open neighbourhood of $(e_1,...,e_k) \in V_{k,n}$.

The map $\xi : U \to S_+^{n-1} \times V_{k-1,n-1}$ defined by
\[ (v_1, ..., v_k) \mapsto (v_1, \psi(v_1)^{-1}(v_2), ..., \psi (v_1)^{-1}
(v_k)) \]
is a homeomorphism. 

\begin{itemize}
\item $\xi$ is well-defined: 
$\psi(v_1)^{-1}$ is an orthogonal matrix such that $\psi(v_1)^{-1} (v_1) = e_1$,
since $\psi(v_1)^{-1}$ is orthogonal and $v_2,...,v_k$ define a $k$-frame in
$(v_1)^\perp$. So $\psi(v_1)^{-1}(v_2), ..., \psi(v_1)(v_k)$ defines a $k$-frame
in $(e_1)^\perp = 0 \otimes \R^{n-1}$.
\item $\xi$ is continues
\item $\xi$ has a continues inverse:
\[  S_+^{n-1} \times C_{k-1,n-1} \to U; \qquad (v, w_1, ..., w_{k-1}) \mapsto
(v, \psi(v)(0,w_1), ..., \psi(v)(0,w_{k-1})), \]
where $(0,w_i) \in \R^n = \R \otimes \R^{n-1}$.
\end{itemize}

Conclution:
The point $(e_1,...,e_n) \in V_{k,n}$ has an open neighbourhood homeomorphic to
$S_+^{n-1} \times V_{k-1,n-1}$, which is a manifold of dimension
\[ d = (n-1) + (n-2) + (n-3) + ... + ( (n-1) - (k-1)), \]
by induction. 
So $(e_1,...,e_k)$ has an open neighbourhood homeomorphic to $\R^d$.

Now let $(v_1, ...,v_k) \in V_{k,n}$ be any point. Complete to an orthogonal
basis
\[ A = (v_1, ..., v_k, v_{k+1}, ..., v_n) \in O(n). \]
Then
\[ A : V_{k,n} \to V_{k,n}; \qquad (w_1, ..., w_k) \mapsto (Aw_1, ..., Aw_k) \]
is a self-homeomorphism of $V_{k,n}$ that sends $(e_1,...,e_k)$ to
$(v_1,...,v_k)$. So also  $(v_1,...,v_k)$ has an open neighbourhood homeomorphic
to $\R^d$.
\end{proof}

\begin{remark}
What we really showed is that the map $V_{k,n} \to S^{n-1}$ defined by
$(v_1,...,v_k) \mapsto v_1$ is a "locally trivial fibre bundle" with fibre
$V_{k-1, n-1}$. 
\end{remark}

\subsection{Complex Steifel manifolds:}

Let 
\begin{align*}
V^\C_{k,n} &= \{ (v_1,...,v_k) \in (\C^n)^k \q: \<{v_i, v_j} 
\begin{cases} 1 \textif i = j \\ 0 \textif i \ne j \end{cases} \}
&= \text{space of (complex) } k\text{-frames in } \C^n
\end{align*}
As in the real case, one shows that $V^\C_{k,n}$ is a compact $d$-manifold,
where
\[ d = (2n -1) + (2n-3) + ... + (2n - 2k + 1) = 2nk - k^2. \]

\subsubsection{special case}

\begin{align*}
V^\C_{1,n} &= \text{unit sphere in } \C^n = S^{2n-1} \\
V^\C_{n-1,n} &\cong \SU(n), \\
V^\C_{n,n} &\cong \U(n)
\end{align*}
Same induction proof, with Gram-Schmidt orthonormalization for hermitian inner
product spaces; In the indutive step, you work over $S_+^{2n-1} = \{ (v_1, ...,
v_n) \in S^{n-1} \q: \Re(v_1) > 0 \}$.


\subsection{Quaternion Stiefel manifolds:}
$V^\H_{k,n}$ defines compact manifolds of dimention
\[ (4n-1) + (4n-5) + ... + (4n-4k+3) = 4nk - k(2k-1). \]



\subsubsection{special case}
\begin{align*}
V^\H_{1,n} &= \text{unit sphere in } \H^n \cong S^{4n-1}, \\
V^\H_{n,n} &= \Sp(n) = \{ A \in M(n\times n, \H) \q: A\dot \bar{A}^T =
\bar{A}^T\cdot A = E_n \}.
\end{align*}

\end{document}
