\documentclass[a4paper,11pt,english]{article}
\usepackage{.style}

%%%%%%% Title %%%%%%%%%%%%%%%%%%%%%%%%%%%
\title{\textbf{Topology II - The mod-2 fundamental class}}
\author{Tor Gjone}

%%%%%%% Definitions %%%%%%%%%%%%%%%%%%%%%



\newcommand{\diagSquare}[8]{
\begin{tikzpicture}[node distance=2cm, auto]
\node (a)              { $ #5 $ };
\node (b) [right of=a] { $ #6 $ };
\node (c) [below of=a] { $ #7 $ };
\node (d) [right of=c] { $ #8 $ };
\draw[-to] (a) to node { $ #1 $ } (b);
\draw[-to] (a) to node { $ #2 $ } (c);
\draw[-to] (b) to node { $ #3 $ } (d);
\draw[-to] (c) to node { $ #4 $ } (d);
\end{tikzpicture}
}


\newcommand{\congto}{\xrightarrow{\cong}}
\newcommand{\incto}[1][]{ \xhookrightarrow{#1} }
\newcommand{\xto}[1]{\xrightarrow{#1}}


% Projective spaces
\def\RP{\mathbb{RP}}
\def\CP{\mathbb{CP}}
\def\HP{\mathbb{HP}}


% Commen Cathegories
\DeclareMathOperator{\Top}{Top}
\DeclareMathOperator{\Ab}{Ab}
\DeclareMathOperator{\Cat}{Cat}
\DeclareMathOperator{\CAT}{CAT}
\DeclareMathOperator{\Mod}{Mod}
\DeclareMathOperator{\Ring}{Ring}
\DeclareMathOperator{\Group}{Group}


% Homological Algebra
\DeclareMathOperator{\Hom}{Hom}
\DeclareMathOperator{\Tor}{Tor}
\DeclareMathOperator{\Ext}{Ext}


% Common Lie Groups 
\DeclareMathOperator{\GL}{GL}
\DeclareMathOperator{\SU}{SU}
\DeclareMathOperator{\U}{U}
\DeclareMathOperator{\Sp}{Sp}

% Maths Operators
\DeclareMathOperator{\id}{id}



\DeclareMathOperator{\Ker}{Ker}
\DeclareMathOperator{\im}{Im}


% Standard setts.
\def \N {\mathbb{N}}
\def \Z {\mathbb{Z}}
\def \Q {\mathbb{Q}}
\def \R {\mathbb{R}}
\def \C {\mathbb{C}}
\def \H {\mathbb{H}}

\def \E {\mathbb{E}}
\def \Z {\mathbb{Z}}
\def \I {\mathbb{I}}
\def \J {\mathbb{J}}

% Vector calculus.
\newcommand{\dif}[3][]{
	\ensuremath{\frac{d^{#1} {#2}}{d {#3}^{#1}}}}
\newcommand{\pdif}[3][]{
	\ensuremath{\frac{\partial^{#1} {#2}}{\partial {#3}^{#1}}}}

% Vectors and matricies.
\newcommand{\mat}[1]{\begin{matrix} #1 \end{matrix}}
\newcommand{\pmat}[1]{\begin{pmatrix} #1 \end{pmatrix}}
\newcommand{\bmat}[1]{\begin{bmatrix} #1 \end{bmatrix}}

% Add space around the argument.
\newcommand{\qq}[1]{\quad#1\quad}
\newcommand{\q}[1]{\:\:#1\:\:}

% Implications
\newcommand{\la}{\ensuremath{\Longleftarrow}}
\newcommand{\ra}{\ensuremath{\Longrightarrow}}
\newcommand{\lra}{\ensuremath{\Longleftrightarrow}}

\newcommand{\pwf}[1]{\begin{cases} #1 \end{cases}}
\newcommand{\tif}{\text{if}}

% Shorthand
\newcommand{\vphi}{\varphi}
\newcommand{\veps}{\varepsilon}

\newcommand{\<}[1]{\langle #1 \rangle}

% Notation
\newcommand{\wddef}[1]{\underline{#1}}


% Maths Operators
\theoremstyle{plain}
\theoremstyle{definition}
\newtheorem{thrm}{Theorem}[section]
\newtheorem{prop}[thrm]{Proposition}
\newtheorem{corol}[thrm]{Corollary}
\newtheorem{lemma}[thrm]{Lemma}

\newtheorem{defn}[thrm]{Definition}
\newtheorem{exmp}[thrm]{Example}
\newtheorem{clame}[thrm]{Clame}


\theoremstyle{remark}
% \newtheorem{remark}[thrm]{\normalfont\large\textit Remark}
\newtheorem{remark}[thrm]{Remark}
\newtheorem{note}[thrm]{Note}


\def\F{\mathbb{F}}

%%%%%%%% Content %%%%%%%%%%%%%%%%%%%%%%%%
\begin{document}
\mmaketitle

% NOT Complete
% Missing proofs 

\begin{thrm}
Let $M$ be a compact connected orientable $n$-manifold. Then for all $x \in M$
and all coefficient groups $A$, the restriction map $r^M_x : H_n(M;A) \to
H_n(M|x;A) \cong A$ is an isomorphism.
\end{thrm}


\begin{proof}

\end{proof}

\begin{note}
The proof showed that the map $H_n(M;\Z) \otimes A \to H_n(M;A)$ from the
universal coefficient theorem is an isomorphism. Hence its cokernel
$\Tor(H_{n-1}(H;\Z), A)$ is zero for all abelian groups $A$. In particular, for
all $k \ge 1$, 
\[  k\text{-torsion in } H_{n-1}(M;\Z) \cong \Tor(H_{n-1}(M;\Z), \Z/k) = 0. \]
So the group $H_{n-1}(M;\Z)$ is torsion free for every compact, connected
orientable $n$-manifold $M$.
\end{note}

Next: "in mod-2 homology there are no orientability issues because $\F_2$ has
only one unit"

\begin{thrm}
Let $M$ be an $n$-manifold and $K$ a compact subset of $M$. Then there is a
unique class $\nu_K \in H_n(M|K; \F_2)$ such that for all $x \in K$, the class
$r^K_x(\mu_K)$ is non-zero, and hence a generator of $H_n(M|x;\F_2) \cong \F_2$.
\end{thrm}

\begin{corol}
Let $M$ be a compact connected $n$-manifold. Then for very $x \in M$ the map 
\[ r^M_x : H_n(M;\F_2) \to H_n(M|x;\F_2) \cong \F_2 \]
is an isomorphism. If $M \ne \emptyset$, then in particular $H_n(M;\F_2) \cong
\F_2$.
\end{corol}

\begin{proof}
Exactly as in the previous video for $\Z$-coefficients and orientable $M$.
\end{proof}

\begin{remark}
Let $M$ be a connected, compact, non-orientable manifold. Then $H_n(M;\Z) = 0$
by the previous video. The universal coefficient theorem provides and
isomorphism  
\[ \F_2 \cong H_n(M; \F_2) \congto \Tor(H_{n-1}(M;\Z), \F_2) = 2\text{-torsion
in } H_{n-1} (M;\Z). \]
\end{remark}

\end{document}
