\documentclass[a4paper,11pt,english]{article}
\usepackage{.style}

%%%%%%% Title %%%%%%%%%%%%%%%%%%%%%%%%%%%
\title{\textbf{Topology II - Homology vanishing above the dimension}}
\author{Tor Gjone}

%%%%%%% Definitions %%%%%%%%%%%%%%%%%%%%%

% Commen Cathegories
\DeclareMathOperator{\Top}{Top}
\DeclareMathOperator{\Ab}{Ab}
\DeclareMathOperator{\Cat}{Cat}
\DeclareMathOperator{\CAT}{CAT}
\DeclareMathOperator{\Mod}{Mod}
\DeclareMathOperator{\Ring}{Ring}
\DeclareMathOperator{\Group}{Group}


% Homological Algebra
\DeclareMathOperator{\Hom}{Hom}
\DeclareMathOperator{\Tor}{Tor}
\DeclareMathOperator{\Ext}{Ext}


% Maths Operators
\DeclareMathOperator{\id}{id}
\DeclareMathOperator{\GL}{GL}

\DeclareMathOperator{\Ker}{Ker}
\DeclareMathOperator{\im}{Im}


% Standard setts.
\def \N {\mathbb{N}}
\def \Z {\mathbb{Z}}
\def \Q {\mathbb{Q}}
\def \R {\mathbb{R}}
\def \C {\mathbb{C}}

\def \E {\mathbb{E}}
\def \Z {\mathbb{Z}}
\def \I {\mathbb{I}}
\def \J {\mathbb{J}}

% Vector calculus.
\newcommand{\dif}[3][]{
	\ensuremath{\frac{d^{#1} {#2}}{d {#3}^{#1}}}}
\newcommand{\pdif}[3][]{
	\ensuremath{\frac{\partial^{#1} {#2}}{\partial {#3}^{#1}}}}

% Vectors and matricies.
\newcommand{\mat}[1]{\begin{matrix} #1 \end{matrix}}
\newcommand{\pmat}[1]{\begin{pmatrix} #1 \end{pmatrix}}
\newcommand{\bmat}[1]{\begin{bmatrix} #1 \end{bmatrix}}

% Add space around the argument.
\newcommand{\qq}[1]{\quad#1\quad}
\newcommand{\q}[1]{\:\:#1\:\:}

% Implications
\newcommand{\la}{\ensuremath{\Longleftarrow}}
\newcommand{\ra}{\ensuremath{\Longrightarrow}}
\newcommand{\lra}{\ensuremath{\Longleftrightarrow}}

\newcommand{\pwf}[1]{\begin{cases} #1 \end{cases}}
\newcommand{\tif}{\text{if}}

% Shorthand
\newcommand{\vphi}{\varphi}
\newcommand{\veps}{\varepsilon}

\newcommand{\<}[1]{\langle #1 \rangle}

% Notation
\newcommand{\wddef}[1]{\underline{#1}}


% Maths Operators
\theoremstyle{plain}
\theoremstyle{definition}
\newtheorem{them}{Theorem}[section]
\newtheorem{prop}[them]{Proposition}
\newtheorem{corol}[them]{Corollary}
\newtheorem{lemma}[them]{Lemma}

\newtheorem{defn}[them]{Definition}
\newtheorem{exmp}[them]{Example}
\newtheorem{clame}[them]{Clame}


\theoremstyle{remark}
% \newtheorem{remark}[them]{\normalfont\large\textit Remark}
\newtheorem{remark}[them]{Remark}


\DeclareMathOperator{\Gr}{Gr}

%%%%%%%% Content %%%%%%%%%%%%%%%%%%%%%%%%
\begin{document}
\mmaketitle

% NOT Complete
% Missing proofs 

All of the manifolds that we explicitly discussed admit CW-structures:

\begin{itemize}
\item $S^n$, $\RP^n$, $\CP^n$ and $\HP^n$,
\item $V_{k,n}$, $\Gr_{k,n}$: CW-structures exists.
\end{itemize}

In all these cases: manifold dimension = CW-dimension

This is no coincidence: Suppose that a compact manifold $M$ admits a
CW-structure; let $x\in M$ be an interior point of a cell of top dimension 
(in the CW-structure). Then the open cell is is an open neighbourhood
homeomorphic to $\R^n$ with $n = $ CW-dimension of $M$. Since the manifold
dimension is intrinsic, $n =$ manifold dimension.

\begin{corol}
Let $M$ be a compact $n$-manifold that admits a CW-structure. Then
$H_i(M; A) \cong H_i^{\text{cell}}(M;A) = 0$ for $i > n$.
\end{corol}

Warning: compact manifolds do not in general admit CW-structures. But every
smooth compact manifold admits a triangulation, and hence a CW-structure.

Notation: For $M$ an $n$-manifold, $A$ an abelian group and $U \subset M$, let $H_i(M|U; A) =
H_i(M, M \setminus U; A)$ denote the local $i$'th homology at $U$ with values in $A$.

\begin{thrm}
Let $M$ be an $n$-manifold, $A$ and abelian group and $K$ a compact subset of
$M$. Then
\begin{enumerate}
\item[(i)] $H_i(M|K; A) = 0$ for $i > n$.
\item[(ii)] A class in $H_n(M|K; A)$ is zero if and only if its
restriction to $H_n(M|x; A)$ is zero for all $x \in K$.
\end{enumerate}
\end{thrm}

\begin{note}
$M$ needs not be compact. But if $M$ is compact, then $K = M$ is allowed and
then both statements refer to absolute homology of $M$.
\end{note}


\begin{proof}
(The proof follows the proof of Lemma A.7 in Appendix A of Milnor-Stasheff's
book "Characteristic Classes")

The proof is done in 6 steps.

\begin{enumerate}
\item[(Step 1)] Consider $M = \R^n$, $K$ is a compact convex non-empty subset.
For every $x \in K$, $K$ can be linearly contracted onto $x$. Let $R > 0$ be
large enough so that $K \subseteq B^{n-1}_R(x) := \{ y \in \R^n \q: |x-y| \le R
\}$. Then the inclusion
\[ S^{-1}_{2R}(x) := \{ y \in \R^n \q: |x-y| = 2R \} \subseteq M\setminus K
\subseteq M \setminus \{ x \}, \]
defines homotopy equivalences. So the induced map $H_i(M|K) \to
H_i(M|x)$ is an isomorphism.

In particular, $H_i(M|K) \cong H_i(M|x) \cong 0$ for $i > n$.

\item[(Step 2)]
Let $M$ be any $n$-manifold, $K = K_1 \cup K_2$ for $K_!, K_2$ compact, suppose
the statements are true for $K_1, K_2$ and $K_1 \cap K_2$. Then the statements
also hold for $K$. We have a long exact Mayer-Vietro's sequence for the local
homology groups. 

\underline{Construction:}
\[ M \setminus (K_1 \cap K_2) = (M\setminus K_1) \cup (M \setminus K_2)
\qq{\text{and}} (M\setminus K_1) \cap (M \setminus K_2) = M \setminus (K_1 \cup
K_2) = M \setminus K. \]
The theorem of small subjections show that the map 
\[ \frac{C_*(M\setminus K_1) \oplus C_*(M\setminus K_2)}{C_*(M\setminus K)}
\incto[\cong] C_*(M \setminus (K_1 \cap K_2)) \]
is an isomorphism of all homology groups.

So the chain map
\[  D := \frac{C_*(M)}{ \left( \frac{C_*(M\setminus K_1) \oplus C_*(M\setminus
K_2)}{C_*(M\setminus K)} \right) } \congto \frac{C_*(M)}{C_*(M\setminus
(K_1 \cap K_2))} \]
is also a quasi-isomorphism.

The so
\end{enumerate}
\end{proof}

\end{document}
